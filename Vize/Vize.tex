\documentclass[a4paper,10pt]{article}

\usepackage[utf8]{inputenc}
\usepackage[czech]{babel} % francais, polish, spanish, ...
\usepackage[T1]{fontenc}

\usepackage{lmodern} % Type1-font for non-english texts and characters

\usepackage{enumitem} % enumerations

\title{Informační systém pro střední školy} % název
\author{Jakub Beránek} % autor
\date{2015} %datum

\begin{document}
	\maketitle
	
	\section{Vize}
	Tento dokument popisuje informační systém pro střední školy, který bude sloužit k evidenci informací o studiu.
	Tento systém je potřebný pro zefektivnění komunikace mezi studenty a učiteli, zajištění centrální správy
	údajů o studiu a zajištění rychlého přístupu k těmto údajům.
	Využívat ho budou učitelé a studenti, pro obě dvě skupiny bude systém poskytovat odlišnou sadu funkcí.
	O správu účtů a konfiguraci celého systému se budou starat administrátoři.
	
	\vspace{5mm} %5mm vertical space
	
	Učitelům bude nabízet správu známek studentů, zadávání jejich absence, přidávání upozornění na blížící se testy a úpravu rozvrhů.
	Studentům bude systém zobrazovat jejich známky, počítat jejich průměry, vypočítávat a upozorňovat na absenci v předmětech,
	zobrazovat rozvrh, suplování a testy zadané učiteli. Administrátoři budou mít možnost spravovat uživatelské účty, předměty, třídy a konfigurovat
	celý systém.
	
	\vspace{5mm} %5mm vertical space
	
	Aplikace bude běžet na serverech školy, uživatelé k ní budou moci přistupovat přes webové rozhraní a mobilní aplikaci.
	Jelikož je potřeba, aby učitelé mohli zadávat známky a testy bez omezení a studenti jej budou využívat dennodenně k získání
	aktuálních informací o studiu, aplikace bude dostupná neustále.
	
	\section{Use case model}
	\subsection{Use case}
	\subsubsection*{UC0: Přihlásit uživatele}
	\begin{description}
		\item[Popis:] Uživatel se přihlásí do systému
		\item[Aktéři:] Uživatel, systém
		\item[Prekondice:] Uživatel není přihlášený
		\item[Scénář:] \hfill \\
				\begin{enumerate}
					\item Systém uživatelovi zobrazí přihlašovací dialog
					\item Uživatel vyplní přihlašovací jméno, heslo a údaj, jestli má být jeho přihlášení zapamatováno
					\item Uživatel odešle údaje
					\item Systém zvaliduje zadané údaje
					\item Systém zkontroluje existenci uživatele a platnost jeho hesla
					\item Systém si zapamatuje přihlášení uživatele
					\item Systém uživatelovi zobrazí úvodní dialog systému
				\end{enumerate}
		\item[Alternativní scénář:] \hfill \\
				\begin{enumerate}
					\setcounter{enumi}{4}
					\setcounter{enumii}{0}
					\begin{enumerate}[label*=\arabic*.,leftmargin=8pt]
						\item
						\begin{enumerate}[label=\alph*.]
							\item Zadané údaje neprošly validací
							\item Systém zobrazí upozornění uživatelovi a vrátí se k bodu 2
						\end{enumerate}
						\setcounter{enumi}{5}
						\setcounter{enumii}{0}
						\item
						\begin{enumerate}[label=\alph*.]
							\item Uživatel neexistuje nebo heslo není platné
							\item Systém zobrazí upozornění uživatelovi a vrátí se k bodu 2
						\end{enumerate}
						\setcounter{enumi}{6}
						\setcounter{enumii}{0}
						\item
						\begin{enumerate}[label=\alph*.]
							\item Uživatel nesouhlasil se zapamatováním přihlášení
							\item Systém jde rovnou k bodu 7
						\end{enumerate}
					\end{enumerate}		
				\end{enumerate}
	\end{description}
	\subsubsection*{UC2: Přidat známky nanečisto}
	\begin{description}
		\item[Popis:] Student vytvoří známku nanečisto a systém vypočte jeho hypotetický průměr s touto známkou
		\item[Aktéři:] Student, systém
		\item[Prekondice:] Student je přihlášený
		\item[Scénář:] \hfill \\
				\begin{enumerate}
					\item Systém studentovi zobrazí seznam jeho známek (viz UC1)
					\item Student zvolí předmět, ke kterému chce přidat známku
					\item Student vyplní hodnotu a váhu známky
					\item Student potvrdí vytvoření známky
					\item Systém zvaliduje zadanou známku
					\item Systém vypočte nový průměr známek studenta
					\item Systém zobrazí nový průměr studentovi
				\end{enumerate}
		\item[Alternativní scénář:] \hfill \\
				\begin{enumerate}
					\setcounter{enumi}{5}
					\setcounter{enumii}{1}
					\begin{enumerate}[label*=\arabic*.,leftmargin=8pt]
						\item
							\begin{enumerate}[label=\alph*.]
								\item Známka zadaná studentem nemá platnou hodnotu
								\item Systém zobrazí upozornění studentovi a vrátí se k bodu 3
							\end{enumerate}
					\end{enumerate}		
				\end{enumerate}
	\end{description}
\end{document}